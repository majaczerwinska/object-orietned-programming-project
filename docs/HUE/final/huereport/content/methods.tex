% !TEX root =  ../report.tex
\section{METHODS}

\subsection{EXPERTS}

To conduct a heuristic usability evaluation 4 expert reviewers were recruited. To ensure an in-depth understanding of the desired user experience we
selected TU Delft students that are currently developing similar applications. The members of Group 46 participated as a group, representing the demographic of a developer team, evaluating together, and submitting a unified report. The 18, 19, and 19-year-old individual reviewers represented consumers, more specifically students in the intended audience.

\subsection{PROCEDURE}
The team of experts was called to evaluate an early version of the application, alongside some mock-ups. The application was presented in 3 stages:
\begin{enumerate}
    \item Via a video call between the members of the developer team and review team, during which the evaluators were given full freedom to explore the product. After analyzing, they were encouraged to pose questions on the logic behind the design and further investigate the user interface.
    \item Three individual reviewers were given full access to the application-under-development, and they each conducted their review under three perspectives: as a first-time user, as a high-end user, and as an admin.
    \item Each of the reviewers was shown the mockups along with an explanation of what the intended functionality is, so they have a better insight into the design.
\end{enumerate}
The application’s developers were overseeing each review process, to ensure the procedure was identical for all reviewers. For any questions that occurred during the process, hints were given only in the case where missing features were preventing a holistic evaluation.

\subsection{HEURISTICS IN USE}
Cited from TU Delft OOPP Team Lecture Slides \cite{lecture-slides}
\begin{enumerate}
    

\item "Visibility of system status: The system should keep users informed about what is going on, with appropriate feedback in good time."

\item "Match between system and the real world: The system should speak the users' language, with words, phrases and concepts familiar at the user (not technical terms)."

\item "User control and freedom: System functions may be chosen by the user by mistake; they need a clearly marked 'emergency exit' without having to go through an extended dialogue."

\item "Consistency and standards: Users should not have to wonder whether different words, situations, or actions mean the same thing."

\item "Error prevention: Even better than good error messages is a careful design which prevents a problem from occurring in the first place."

\item "Recognition rather than recall: Try to make objects, actions, and options visible. The user should not have to mentally hold information and transfer it to another part of the interface. Instructions for using the system should be visible or easily retrievable whenever appropriate."

\item "Flexibility and efficiency of use: Accelerators (such as a function keys or macros) and senior buyer of the novice user may often speed up the interaction of for the expert user. Thus the system that can cater to both inexperienced and experienced users."

\item "Aesthetic and minimalist design: Dialogues or other interface items should not contain information which is irrelevant or rarely needed. All information on the screen competes with the relevant units of information and diminishes their relative visibility."

\item "Help users recognize, diagnose, and recover from errors: Helping users recognize, diagnose, and recover from errors. Error messages should be expressed in plain language, precisely indicate the problem, and constructively suggest a solution."

\item "Help and documentation: Although the system should be able to be used without documentation, it may be necessary to provide help of some form. This information should be easy to search, focused on the user’s task (context sensitive), list the  steps to be carried out, and be brief and to the point."

\end{enumerate}


\subsection{MEASURES}

During the video call, the experts were able to observe the scenes as well as the flow of the application. To measure user satisfaction with the application’s experience, the reviewing team simulated a reasonable use of the app, taking notes of the above-stated heuristic standards as seen in the app. The meeting was recorded to allow the team to conduct a thorough examination, summarised in a document that highlights the discovered flaws. Reviewers reported issues in a specific format, prepared with a consultation to the video slides on Heuristic Usability Evaluation provided by TU Delft. It consisted of 5 points, namely:

\noindent\( \circ \) problem description 

\noindent\( \circ \) possible difficulties

\noindent\( \circ \) specific context of the problem 

\noindent\( \circ \) assumed causes

\noindent\( \circ \) the violated heuristic standards (from the aforementioned list)

The individual reviewers got to use the application under supervision, taking notes using the aforementioned format. The scenarios were:

\noindent\( \circ \) New user, without any information on how the application works, attempts to find their way around the app and explore functionality, noting down what makes sense and what does not.

\noindent\( \circ \) Advanced user, with knowledge of all tools of the app (keyboard shortcuts, special functions), examines how efficient the workflow is.

\noindent\( \circ \) System Admin, evaluates how easy moderating a server is.

