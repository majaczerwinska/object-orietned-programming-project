% !TEX root =  ../report.tex
\section{RESULTS}
\indent Several issues with the current state of the application, regarding both its design and functionality, have been discovered. The reviews received from the evaluators were used to describe problems, and later on, improvement areas.
\newline

\subsection*{Problem 1. Overview Clutter}
The board page is too full, all the lists and buttons make the space very tight. It is difficult to navigate through the board without feeling overwhelmed.
\newline
Related Heuristic: 8
\newline
\indent In the board overview page, the app currently has many functional elements in the same space, which can over-whelm a novice user. It might lead to confusion and degraded overall user experience.
\newline
"[it is] Difficult to navigate and use the board effectively without it seeming overbearing"\cite{team-review}
\newline
\indent The resolution of this problem is a high priority since the user will likely spend most of their time on the board overview page. As the heart of the application, it is critical that the user experience is best here.
\newline

\subsection*{Problem 2.}
The purpose of board keys in the app is inadequately explained by the time the user is required to enter one, and if they don’t understand how board keys work they will miss out on the app’s features.
\newline
Related Heuristics: 4, 6, 10
\newline
\indent About 50\% of the evaluators found that the application’s approach to joining a board using a board key may prove difficult for users unfamiliar with the system. It is not clear how board keys work.
 
\indent The frequency of this problem alongside the fact that it is a crucial feature of the app will prioritise its resolution.
\newline

\subsection*{Problem 3.}
The Landing Page does not reflect the actual application's use and is unclear what each button is for.
\newline
Related Heuristics: 8, 10
\newline
\indent The Landing Page is confusing for new users since it is nowhere specified what the options "public", "team" and "private" actually mean and do. Even though few reviewers reported this, the problem is severe, since it affects the ease of use of the app, because the Landing Page is the entry point of the application.
\newline

\subsection*{Problem 4.}
The application's color customisation options are user-unfriendly.
\newline
Related Heuristics: 6, 7
\newline
\indent
The app allows users to customize their board theme but lacks a preview option for selected colors. Additionally, there is no default color palette, requiring users to remember their board’s colors when making selections such as choosing a color for a new tag.
\newline
\indent This issue is not impeding the app’s functionality, but will also be taken into account and eventually resolved.
\newline

\subsection*{Problem 5.}
Adding a new list to the board on the overview page is unintuitive, and the related button is hard to reach.
\newline
Related Heuristics: 6,8
\newline
\indent The reviewers found the position of the add list button to be inappropriate (located at the end of the horizontally scrollable view of lists, on the right side of the board). 
\newline
\indent This issue is of low severity, and can be easily resolved.
\newline

\subsection*{Problem 6. }
Incomprehensive password requirement field on board creation page.
\newline
Related Heuristics: 8
\newline
\indent The Board Creation Overview page includes a pass-word field and a "password required" radio button, but leaving the password field empty should imply no password. The radio button was added for extra precaution, but it’s deemed confusing. The asterisk next to the password field incorrectly implies it’s required.
\newline
\indent Password protection is essential, it is important that users can efficiently use it. The issue is of moderate severity, but simple to fix.
\newline

\subsection*{Problem 7.}
Redundant "Exit" Button on Server Selection window.
\newline
Related Heuristics: 8
\newline
\indent Reviewers found the exit button on the Server Selection window redundant and cluttering, as users can simply close the window to exit the app. The button takes up space without providing any additional functionality.
\newline
\indent The problem is of low severity and it is easy to solve since the button can just be deleted and the rest of the screen can be rearranged to adhere to a minimalist design.