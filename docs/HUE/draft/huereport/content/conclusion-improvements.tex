% !TEX root =  ../report.tex
\section{CONCLUSIONS \& IMPROVEMENTS}
\subsection{IMPROVEMENTS}
After conducting the heuristic usability evaluation, we identified areas for improvement in the design and user interface of the task management application. The following sections describe the changes we recommend to enhance the overall user experience.
\newline
Improvement Area 1: Board Overview Scene
\newline
\indent The current design of the board overview scene suffers from two severe issues: clutter and poor discoverability of the add list button. To address these problems, the following changes have been proposed:
\newline
\indent First, the add list button will be moved to the board sidebar, making it more accessible and prominent.
\newline 
Second, the overall design will be simplified by removing card details from the board overview scene and creating a new card overview scene. Users can access the card overview by double-clicking on the card name. This separation of information between the board and card overview scenes will result in a more minimalist and user-friendly application.
\newline
\newline
Improvement Area 2.
\indent This area of the heuristic usability evaluation report identified two main issues with the board creation menu: the board key is confusing for new users, and the password system for the boards is inconsistent. 
\newline
\indent One proposed solution to the first issue is to change the text that refers to the board key to “Use this key to share your board with others” and modify the placeholder text for the board key field to be more specific, such as “Type the key for the board you want to join.” 
\newline
\indent To address the second issue, the recommended improvement is to add a drop-down field where the user can enter the password for the board only if the radio button is checked, which will make it easier to understand for new users.
\newline
\newline
Improvement Area 3.
\indent The Landing Page may be difficult to navigate for users without prior context or familiarity with the application.
\newline
\indent However, this issue can be addressed by either providing a brief written tutorial on how to use the app or by reviewing the design of the starting page. If the latter approach is taken, the buttons on the Landing Page could be styled with hover effects that provide more information on where each command will redirect the user.
\newline
\newline
Improvement Area 4.
\indent The color customization feature is difficult to use. One possible solution is to include a few buttons with commonly used colors and a color picker wheel that can be implemented next to the color text field, making it easier to use.
\newline
\newline
To prioritize which problems should be addressed first, we can evaluate them based on their severity and frequency. A matrix can be created with severity plotted horizontally and frequency plotted vertically. This will provide an overview of which problems should be prioritized for fixing. The problems in the top left corner of the matrix, with high severity and frequency, should be addressed as soon as possible, while those in the bottom left corner, with low severity and frequency, can be considered last. The matrix with the problem numbers can be found below.
\newline
\newline
\begin{tabular}{ |p{1.5cm}|p{1.6cm}|p{1cm}| p{1.2cm}|p{.9cm}|p{1.3cm}|}
\hline
\multicolumn{6}{|c|}{Prioritizing severity matrix} \\
\hline
Severity\newline  Frequency& Catastrophic&Critical&Moderate &Minor&Negligible\\
\hline
Frequent &&\centering 2\newline \centering 3& \centering 1&& \\
\hline
Probable &&&&& \\
\hline
Occasional &&&& & \\
\hline
Remote  &&& \centering 6& \centering 5&  \\
\hline
Improbable &&&&  \centering 4&   7 \\
\hline
\end{tabular}

\subsection{Conclusion}

In conclusion, based on the findings from the conducted Heuristic Usability Evaluation, the following measures will be taken into consideration in order to enhance the usability experience:
\begin{itemize}
    \item The overall application design will be remodeled so that it could overcome its drawbacks.
    \item The new plans will be given to the developers to be implemented in the application code.
\end{itemize}
The purpose of the aforementioned improvements is to boost the user experience and the efficiency of the application.       