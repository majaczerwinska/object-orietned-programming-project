% !TEX root =  ../report.tex
\section{METHODS}

\subsection{EXPERTS}

To conduct a heuristic usability evaluation members of group 46 were recruited. The group consists of 6 TU Delft students that are currently developing a similar application that guarantees their in-depth understanding of the desired user experience and an objective view of the user interface.

\subsection{PROCEDURE}
The team of experts was called to evaluate an early version of the application, alongside some mock-ups. Intended to showcase the stylistic decisions that will be implemented at a later stage of production, the mock-ups served as a visually complete overview of the user experience. This prototype of the application was presented via a video call between the members of the two teams, during which the evaluators were given full freedom to explore the product. After analyzing, they were encouraged to pose questions on the logic behind the design and further investigate the user interface.

\subsection{HEURISTICS IN USE}
Cited from TU Delft OOPP Team Lecture Slides \cite{lecture-slides}
\begin{enumerate}
    

\item "Visibility of system status: The system should keep users informed about what is going on, with appropriate feedback in good time."

\item "Match between system and the real world: The system should speak the users' language, with words, phrases and concepts familiar at the user (not technical terms)."

\item "User control and freedom: System functions may be chosen by the user by mistake; they need a clearly marked 'emergency exit' without having to go through an extended dialogue."

\item "Consistency and standards: Users should not have to wonder whether different words, situations, or actions mean the same thing."

\item "Error prevention: Even better than good error messages is a careful design which prevents a problem from occurring in the first place."

\item "Recognition rather than recall: Try to make objects, actions, and options visible. The user should not have to mentally hold information and transfer it to another part of the interface. Instructions for using the system should be visible or easily retrievable whenever appropriate."

\item "Flexibility and efficiency of use: Accelerators (such as a function keys or macros) and senior buyer of the novice user may often speed up the interaction of for the expert user. Thus the system that can cater to both inexperienced and experienced users."

\item "Aesthetic and minimalist design: Dialogues or other interface items should not contain information which is irrelevant or rarely needed. All information on the screen competes with the relevant units of information and diminishes their relative visibility."

\item "Help users recognize, diagnose, and recover from errors: Helping users recognize, diagnose, and recover from errors. Error messages should be expressed in plain language, precisely indicate the problem, and constructively suggest a solution."

\item "Help and documentation: Although the system should be able to be used without documentation, it may be necessary to provide help of some form. This information should be easy to search, focused on the user’s task (context sensitive), list the  steps to be carried out, and be brief and to the point."

\end{enumerate}


\subsection{MEASURES}

During the video call, the experts were able to observe the scenes as well as the natural flow of the application.
Intending to measure user satisfaction with the application's experience, the team initially simulated a reasonable use of the app, taking notes of the above-stated heuristic standards as seen in the app.
The meeting was recorded for future use, thus allowing the team to conduct a thorough examination, summarised in a document that highlights the discovered flaws.
The issues were reported in a specific format that was prepared by us with a consultation to the video slides on Heuristic Usability Evaluation provided by TU Delft. It consisted of 5 points, namely:
\begin{itemize}
    \item problem description 
    \item possible difficulties
    \item specific context of the problem 
    \item assumed causes
    \item the violated heuristic standards (from the aforementioned list)
\end{itemize}

